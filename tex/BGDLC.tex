\todo{what is live-cycle development}

It is easy to find many software development life-cycle definition in the literature \cite{IEEE_Std_1074_1997,sommerville2007software,abrahamsson2003new,IEEE_Std_12207_2008,chatterjee2008software}, but defining a life cycle for agile software development is more difficult as agile in essence does not follow a rigorous process, as it can be seen in the Manifesto for agile software development \cite{alliance2001agile}. By the manifesto it is possible to see that processes, tools, comprehensive documentation, negotiation of contracts, and following a plan is not the most important, so it is possible to understand why it is difficult to find a well defined life cycle that synthesizes the agile way of developing software. 

On the other hand, there are some processes models to create \acrfull{SOA} applications as it is possible to see in \citet{Lane2011} such as Collaborative modeling (\todo{S22}), Semantic modeling (\todo{S37}), SOMA \cite{ArsanjaniSOMA2006}, Context aware (\todo{Context aware}), Web based (\todo{Web based}) and so on. In order to classify the processes into groups of related processes the authors used reciprocal translation \cite{noblit1988meta} and they synthesises the result in to a service development process meta-model, that present the following phases: analysis and design, construction and testing, deployment and provisioning and, execution and monitoring. This meta-model is similar to a map from a mapping study as outlined by \cite{budgen2008using,Petersen2008}.
