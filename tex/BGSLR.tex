One instrument used to synthesizing evidence in some research area is the \acrfull{SLR} \cite{Petersen2015}. \acrshort{SLR} is a methodologically rigorous review of research results and its main aim is aggregate all existing evidence on a research question and it is initially used in medicine researchers \cite{Kitchenham2009}. \citet{Kitchenham2004} presented the guideline for performing \acrshort{SLR} in Software Engineering that was derived from existing guidelines used by medical researches. According to \citet{Kitchenham2004}:

\textit{``A systematic literature review is a means of identifying, evaluating and interpreting all available research relevant to a particular research question, or topic area, or phenomenon of interest. Individual studies contributing to a systematic review are called primary studies; a systematic review is a form a secondary study.''}

According to \citet{Kitchenham2004}, the \acrshort{SLR} has more scientific value than a simple literature review due to the methodological rigor of \acrshort{SLR}. The reasons to performing a \acrshort{SLR} are to summarize the existing evidence concerning a treatment or technology, to identify any gaps in current research in order to suggest areas for further investigation and to provide a framework/background in order to appropriately position new research activities. 