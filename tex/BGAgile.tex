A light method for software development can be understood as a method that the needs and goals of the stakeholders drive the development and enable the rapid and flexible development of software. The processes and tools itself are less important than deliverables, people, developers and stakeholders, communication, and the ability to adapt to change. This type of method has to be iterative, incremental, cooperative and adaptable to changes in environments and requirements \cite{Abdelouhab2014,Chehili2013,Farroha2011}. The search for this type of software development method led to the “agile methods” that were stated in the Agile Manifesto, published in 2001 \cite{alliance2001agile} where seventeen developers and consultants representatives of light methods for software development met to discuss better ways of developing software.

There are many methods of software development that claim to be agile development methods such as SCRUM, \acrfull{RUP}, \acrfull{DSDM}, \acrfull{ASD}, \acrfull{ISD}, Lean, Crystal Clear, \acrfull{XP}, \acrfull{FDD}, \acrfull{PP}, and so on \cite{Carvalho2013,INAYAT2015915,Highsmith2001Agile,DYBA2008833,VallonSLRAgile2018,Timperi2004}. Each of them has its own phases and a life cycle, and some methods do not even define well the stages of development. The following are some characteristics of each of the main of these agile methodologies, according to \citet{Timperi2004}:
\begin{itemize}
\item \acrfull{FDD} is based on several best practices and it emphasizes design and building activities. The requirements are captured first by constructing a domain object model and using it as a basis for requirements elicitation and inspections are the main quality assurance practice but testing is also mandatory.
\item \acrfull{XP} is based on short iterations and incremental development with constant feedback from both the customer and other developers and most of the practices of XP are aimed at quality assurance and in particular getting timely feedback.
\item \acrfull{DSDM} relies heavily on prototyping in most development activities and proposes a pragmatic view to quality; the emphasis is on early validation while technical quality can be sacrificed.
\item SCRUM is based on empirical management and it does not state engineering practices and consists of development in iterations called sprints. The requirements are captured in prioritized order in a product backlog and in a sprint backlog for the current sprint. Daily management is handled by scrum meetings in which participants answer three questions regarding what they have done since the last scrum meeting, what they will do between now and the next scrum meeting, and what problems they have.
\end{itemize}



